The colourful $k$-center problem is a generalisation of the Robust $k$-center problem (\cite{bandyapadhyay_constant_2019}). It introduces $\emph{c}$ new parameters which represent the demographic that each client belongs to. The set of clients V is partitioned into $\emph{c}$ classes, $V=\{V_1, V_2, ... V_{\emph{c}}\}$, and for each class $\emph{i}$ at least $1\leq t_i \leq |V_i|$ clients must be covered. Note that when $\emph{c}=1$, we have the robust $k$-center problem.

When defining the robust $k$-center problem, we introduced the parcel delivery service analogy which is arguably an application which is arguably not detrimental when we exclude certain clients in our optimal solution. However imagine if we take a similar analogy but apply it to an essential and time sensitive service such as facility location for hospitals with ambulances. The robust $k$-center problem will be biased towards denser clusters and neglect to cover outliers. If we have some prior knowledge of the clients, for example suppose people over 60 tend to be more medically vulnerable and live on the outskirts of the city, solving the robust $k$-center problem will neglect people over 60. Therefore the hospitals will be located further away from a group that is more likely to require the ambulance service. Instead if we apply the colourful $k$-center problem and partition the clients into $\emph{c}=2$ groups, we can specify that we would like to cover 100\% of clients aged over 60 and 80\% of clients under 60. Solving the colourful $k$-center problem under these constraints will produce an allocation of hospitals which take into account of outliers without excluding a particular sub-population.

This can be formally stated in the following:

\textbox{Given a graph $G=(V, E)$ on a set of $n$ vertices $V = \{v_1,..., v_n\}$, colour classes $1,...,c$, where the vertices are partitioned into the colour classes $V_1, ...,V_c$, a maximum number of centers $k$ and we are given a minimum coverage constraint corresponding to each colour class $t_1,...,t_c$ (where $1\leq k\leq n$ and $\forall _{i\in\llbracket 1\mathrel{{.}\,{.}}\nobreak c\rrbracket}$ $1\leq t_i\leq n$): 

\vspace{0.2cm}We want to find a subset of centers $C\subseteq V$ which covers a subset of vertices $Z\subseteq V$, such that when Z is partitioned into $c$ colour classes $Z=\{Z_1,...,Z_c\}$ the coverage constraint $\forall _{i\in\llbracket 1\mathrel{{.}\,{.}}\nobreak c\rrbracket}$ $Z_i\geq t_i$ is satisfied, while minimising the objective function: $max_{v\in Z}min_{c\in C}dist(v, c)$.}

The work by \textcite{bandyapadhyay_constant_2019}, showed a pseudo-approximation algorithm with a constant approximation factor. The pseudo approximation means that while the cost of the solution is bounded by the approximation factor, one or more of the constraints is relaxed. In the case of the constant factor pseudo-approximation algorithm, the number of centers in a solution will be at most $k+1$ (rather than adhering to the original constraint of using at most $k$ centers). Details about the implications of the constant approximation factor and the implementation of their algorithm is in detailed in \cref{section:colourful_k_center}. In the latter part of their paper, they proposed a method to form a true approximation for the colourful $k$-center problem limited to the two dimensional euclidean plane, by introducing another algorithm which exploits some geometric relationships between the Colourful $k$-center problem and the Exact Perfect Matching problem. This was combined with the constant factor pseudo-approximation algorithm to produced a ($17+\epsilon$)-approximation where $\epsilon\geq 0$ (\cite{bandyapadhyay_constant_2019}). However as noted by \textcite{anegg_technique_2020}, this extension is quite complicated and is limited to only 2D euclidean spaces which is the reason why we do not consider it in our implementation.

In a recent paper, a 3-approximation algorithm was proposed by \textcite{jia_fair_2020}. They noticed the relation between the Colourful $k$-center and the subset sum problems; using that, they used the constant factor pseudo-approximation algorithm (\cite{bandyapadhyay_constant_2019}) as a subroutine and combined with a dynamic program which solves the subset sum problem. Their algorithm, unlike the true approximation algorithm by \textcite{bandyapadhyay_constant_2019}, does not have restrictions on the metric space that the graph is situated in.

Independently, \textcite{anegg_technique_2020} proposed a 4-approximation algorithm in the same year. They also make use of the constant factor pseudo-approximation algorithm (\cite{bandyapadhyay_constant_2019}) as a subroutine but they deal with the fact that the pseudo approximation may open up one too many centers in a different way. Their method takes advantage of the outputs of the pseudo approximation algorithm, which are the centers $C$ (recall that $|C|\leq k+1$) and radius $r$: they use a combination of dynamic programming and the ellipsoid method to either pick $k$ centers from $C$ then expand the radius to $4r$, or in the case where no such valid solution exists open centers from outside of $V\setminus C$ and choose the remaining centers from $C$.

The algorithms described by \textcite{jia_fair_2020} and \textcite{anegg_technique_2020} are the first true approximations for the Colourful $k$-center and were presented in IPCO 2020 virtual conference (Integer Programming and Combinatorial Optimization). 