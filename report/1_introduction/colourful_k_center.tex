The colourful $k$-center problem (\cite{bandyapadhyay_constant_2019}) is a generalisation of the Robust $k$-center problem. It introduces $\emph{c}$ new parameters which represent the demographic that each client belongs to. The set of clients V is partitioned into $\emph{c}$ classes, $V=\{V_1,... V_{\emph{c}}\}$, and for each class $\emph{i}$ at least $1\leq t_i \leq |V_i|$ clients must be covered. Note when $\emph{c}=1$, we have the robust $k$-center problem.

In the parcel delivery service analogy, excluding clients is not detrimental. However, imagine performing facility location for an essential service such as hospital location for ambulances. If we have some prior knowledge of the clients, for example suppose people aged over 60 tend to be more medically vulnerable and live on the outskirts of the city. Solving the robust $k$-center for this scenario will neglect people aged over 60, since the robust $k$-center biased towards denser clusters. Therefore the hospitals will be located further away from a group that is more likely to require the ambulance service. Instead if we apply the colourful $k$-center problem and partition the clients into $\emph{c}=2$ groups (aged over and under 60), we can specify that we would like to cover 100\% of clients aged over 60 and 80\% of clients under 60. Solving the colourful $k$-center problem under these constraints will produce an allocation of hospitals which corrects for the bias.

This can be formally stated in the following:

\textbox{Given a graph $G=(V, E)$ on a set of $n$ vertices $V = \{v_1,..., v_n\}$, colour classes $1,...,c$, where the vertices are partitioned into the colour classes $V_1, ...,V_c$, a maximum number of centers $k$ and we are given a minimum coverage constraint corresponding to each colour class $t_1,...,t_c$ (where $1\leq k\leq n$ and $\forall _{i\in\llbracket 1\mathrel{{.}\,{.}}\nobreak c\rrbracket}$ $1\leq t_i\leq n$): 

\vspace{0.2cm}We want to find a subset of centers $C\subseteq V$ which covers a subset of vertices $Z\subseteq V$, such that when Z is partitioned into $c$ colour classes $Z=\{Z_1,...,Z_c\}$ the coverage constraint $\forall _{i\in\llbracket 1\mathrel{{.}\,{.}}\nobreak c\rrbracket}$ $Z_i\geq t_i$ is satisfied, while minimising the objective function: $max_{v\in Z}min_{c\in C}dist(v, c)$.}

\citeauthor{bandyapadhyay_constant_2019}, proposed a pseudo $2$-\gls{approx_algs} algorithm for the colourful $k$-center. The \gls{pseudo_approximation} means that while the solution cost is bounded by the approximation factor, at least one of the constraints is violated. In the case of their algorithm, the number of centers in a solution will be at most $k+1$ (rather than adhering to the original constraint of $|C|\leq k$). In the latter part of their paper, they proposed another algorithm which exploits some geometric relationships between the colourful $k$-center problem and the Exact Perfect Matching problem. This was combined with the pseudo $2$-approximation algorithm to produce a true $O(1)$-approximation algorithm.

In a recent paper, a 3-approximation algorithm was proposed by \textcite{jia_fair_2020}, they noticed a relation between the Colourful $k$-center and the subset sum problem; they used the pseudo $2$-approximation algorithm as a subroutine and combined with a dynamic program which solves the subset sum problem to create their algorithm.

Independently, \textcite{anegg_technique_2020} proposed an algorithm for the colourful $k$-center. Their method also uses the pseudo $2$-approximation algorithm as a subroutine, they use a combination of \gls{dynamic_programming} and the \gls{ellipsoid_method} to produce a 4-approximation algorithm.

The algorithms proposed by \textcite{jia_fair_2020} and \textcite{anegg_technique_2020} do not have restrictions on the metric space, unlike the $O(1)$-approximation algorithm. These are the first true approximations for the colourful $k$-center and were presented in \acrshort{ipco} 2020 virtual conference. 