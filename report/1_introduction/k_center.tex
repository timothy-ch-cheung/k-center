The problem setting for the $k$-center is an undirected weighted graph $G=(V,E)$ on $n$ vertices. Given two vertices $v_1, v_2$ the weight of the edge connecting them is given by the function $dist(v_1,v_2)\rightarrow\mathbb{R}_{\ge 0}$. The distance function must satisfy triangular inequality; given any three edges of a graph forming a triangle, the sum of any two sides is less than or equal to the length of the remaining side.

\textcite{hakimi_optimum_1964} first introduced the absolute center and the vertex center. The goal of the vertex center is to find a vertex $c$ which minimises the maximum $dist(c, v_i)$ for all vertices $v_i\in V$. In contrast, the absolute center considers the case where any point along an edge can become a center, the goal is to find a point $c$ which minimises the cost of the maximum shortest path from $c$ to a vertex. It should be noted that in the remainder of this paper, when we refer to k-center problems, we are referring to the vertex center rather than the absolute center.

Further work generalised the absolute and vertex center problems, by considering the allocation of $k$ centers (\cite{hakimi_optimum_1965}). The $k$-center problem can be thought of as a facility location problem, where we want to allocate $k$ facilities in order to best service the $n$ clients. The clients $N$ and potential location for facilities $M$ are subsets of $V$, such that $N\subseteq V$ and $M\subseteq V$. We consider the scenario where every vertex can either be a client or a facility, where $M=N=V$. The $k$-center problem can therefore be defined more formally as follows:

\textbox{Given a graph $G=(V, E)$ on a set of $n$ vertices $V = \{v_1,..., v_n\}$ and a parameter $k$ specifying the maximum number of centers (where $1\leq k\leq n$): 

\vspace{0.2cm}We want to find a subset of centers $C\subseteq V$ such that $|C|\leq k$ which minimises the objective function: $max_{v\in V}min_{c\in C}dist(v, c)$}

An alternative way to describe this problem would be to imagine a city where we would like to build warehouses for a parcel delivery service. Given a destination for a parcel, the parcel will depart from the warehouse nearest to the destination. The city would like to build these warehouses in the optimal locations such that the largest distance a delivery driver must travel is minimised. Now consider that the city has a limited budget and they can only build $k$ warehouses. This analogy is now equivalent to the $k$-center problem.

Later work showed while solving the vertex 1-center problem is a trivial process, solving the $k$-center is NP-hard (\cite{kariv_algorithmic_1979}). Their worked focused on a simplified version of the $k$-center problem, where the graph has a \gls{tree} structure. \textcite{kariv_algorithmic_1979} reported a polynomial-time algorithm which solves the $k$-center problem optimally on trees.

As the $k$-center is NP-hard, there exists no algorithm which gives an optimal solution for graphs of arbitrary structure in polynomial-time. Therefore when designing algorithms for NP-hard problems, we either aim to give an optimal solution in exponential time or potentially non-optimal solutions in polynomial time; we focus on the latter as we are interested in solutions for larger data sets. Throughout this paper, when we refer to a "solution" for the $k$-center problem (and its variants) we are referring to a valid solution which may or may not have the optimal cost. The primary algorithm paradigms applied to the $k$-center problem are \gls{approx_algs} algorithms, \gls{randomised_algs} algorithms, \glspl{metaheuristic} and \gls{local_search} techniques.

Several \gls{approx_algs} algorithms have been proposed to solve the $k$-center problem, the most widely cited are is the \emph{Gon} algorithm (\cite{gonzalez_clustering_1985}) and the \emph{HS} algorithm (\cite{hochbaum_best_1985}) which both produce 2-approximations. The \emph{Gon} algorithm is based on a greedy \gls{heuristic} and the \emph{HS} algorithm is based on a binary search. Both papers independently proved that the $k$-center problem is not only NP-hard to solve optimally, but it is also NP-hard to approximate to for any \gls{approx_algs} factor less than two.

One of the more influential \glspl{metaheuristic} for \gls{combinatorial_optimisation} is Variable Neighbourhood Search, or \acrshort{vns} (\cite{mladenovic_variable_1997}). The basic concept is to search the current neighbourhood of solutions, and then switch neighbourhood if better solutions than the incumbent is found. The \acrshort{vns} metaheuristic was later applied to the $k$-center problem by \textcite{mladenovic_solving_2003}. To efficiently implement this, data structures were designed to store the closest and second closest facility to each client. Their application of the \acrshort{vns} metaheuristic resulted in an algorithm with $\mathcal{O}(n^{2})$ time complexity. They evaluated \acrshort{vns} on OR-LIB, a standard data set originally used for $k$-median problems (\cite{beasley_note_1985}), later adapted for the $k$-center. Their results showed that the \acrshort{vns} algorithm outperformed the \emph{Gon} and \emph{HS} \gls{approx_algs} algorithms.

Later publications built upon the VNS algorithm, the \acrshort{vns} neighbourhood structures were used in the \acrshort{pbs} algorithm (\cite{pullan_memetic_2008}). \acrshort{pbs} is a memetic algorithm, which is an extension of a \gls{genetic_algs} algorithm that employs a local search. Their work showed the ability to converge to the optimal cost in OR-LIB where \acrshort{vns} could not.

To the best of our knowledge, the most recent metaheuristic literature for the $k$-center is the Plateau Surfer algorithm (\cite{battiti_new_2017}). Their work builds upon the observations by \citeauthor{mladenovic_solving_2003} that $k$-center solutions may have multiple vertices which have a distance it their nearest center equal to the solution cost. They observed that these points could indicate local minima in the search, and proposed algorithms to escape them. They reported that their method produced near optimal solutions faster than state of the art algorithms.

While both the \acrshort{pbs} and Plateau Surfer were reported to have better performance than \acrshort{vns}, \citeauthor{battiti_new_2017} did not compare the \acrshort{pbs} algorithm to their Plateau surfer algorithm; we are unable to compare them from their results as the authors used different testing methodologies. In our study we will implement both and compare their performance.