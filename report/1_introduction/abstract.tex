The k-center problem is a classical \gls{combinatorial_optimisation} problem with applications in machine learning and real world applications such as warehouse location. The $k$-center problem aims to allocate at most $k$ facilities to n clients such that the distance the furthest client has to travel to their nearest center is minimised. In recent years, there has been a increase interest in fairness constraints on classical theoretical problems. This led to the introduction of the colourful $k$-center problem. Given a data set with a split demographic, the colourful $k$-center adds constraints where a minimum number of individuals from each demographic must receive service from the set of centers. We are interested in fairness constraints because the classical algorithms may have inbuilt biases which lead them to exclude service from certain groups.

This paper aims evaluate the performance of approximation algorithms and metaheuristic approaches on the $k$-center and colourful $k$-center problems. To date, the colourful $k$-center has only been studied from a theoretical viewpoint, therefore we contribute two new data sets for the colourful $k$-center problem to study it from an empirical viewpoint. The first is GOWALLA (a 3D data set based on real world data) and the second is SYNTHETIC (a 2D artificial data set with known optimal solutions). Finally, we present a \gls{memetic_algs} algorithm, a \gls{genetic_algs} algorithm which utilises a local search, which solves the colourful $k$-center problem. We observed that our algorithm produces solutions with an average of 48.6\% lower cost than existing algorithms.

Furthermore, we develop a web tool for algorithm visualisation. It aims to educate the user about $k$-center problems, visualise solutions and provides a step by step high level overview of how the algorithms work.