The k-center problem is a classical combinatorial optimization problem with applications in machine learning and real world applications such as warehouse location. The $k$-center problem aims to allocate at most $k$ facilities to n clients such that the distance that the furthest client has to travel to their nearest center is minimised. In recent years, there has been a increase interest in fairness constraints on classical problems. This led to the introduction of the colourful $k$-center problem which, given a data set with a split demographic, adds constraints where a minimum number of individuals from each demographic must receive service from a center. We are interested in the $k$-center problem with fairness constraints because the classical algorithms may have inbuilt biases which lead them to exclude service from certain demographics.

This paper aims evaluate the performance of approximation algorithms against other methods such as local search and genetic algorithms (which do not provide any quality guarantee of the solution) on the classical $k$-center problem. We implement these algorithms and compare their performance using empirical methods. We also implement an approximation algorithm designed to solve the colourful variant of the problem. Following this, we present a memetic algorithm which solves the colourful $k$-center problem and we compare its performance to existing algorithms.

In addition we develop a tool with two aims; to visualise solutions given by each algorithm and secondly to provide a step by step high level overview of how they work.