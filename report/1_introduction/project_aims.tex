Our primary aim is to implement various algorithms which solve the $k$-center problem. From implementing these algorithms, we aim to adapt the techniques used to propose an algorithm to solve colourful $k$-center problem which produces better results than the algorithm presented by \textcite{bandyapadhyay_constant_2019}. To perform a robust comparison, we use empirical methods with a variety of standard data sets to compare algorithms based on quality of solution and computational resources used.

Furthermore, from our literature review we noticed that both theoretical papers and papers that presented empirical evidence lacked publicly available implementations of algorithms described. Therefore, as a secondary objective, we were motivated to provide clarity on how these algorithms work in action. To do this we aim to develop a web application which has two aims: The first is to provide implementations of the algorithms presented, where the user can observe the behaviour of various algorithms on different data sets. The second is to break down each individual algorithm so that the user can see the state of the solution in intermediate steps, with objective of improving their understanding on how it arrives that the solution. Through our web application we provide interactive visualisations for the traditional and colourful variant of the $k$-center problem.