The robust $k$-center problem (also known as the $k$-center problem with outliers), introduced by \textcite{charikar_algorithms_2001}, extends $k$-center problem by introducing a constraint \(p\) (\(1 \leq p \leq n \)) which specifies the minimum number of clients that must be serviced the solution. This formulation of the problem is more applicable to real scenarios, for example in our parcel delivery service analogy (\cref{section:k_center}), there may be houses on the outskirts of the city which are not as profitable to deliver to (since the clients are less concentrated in that area) therefore we may want to exclude them when considering the locations to build the warehouses. Therefore the city can change their requirements and ask to build $k$ warehouses such that 80\% of the city's population lives in a reasonable distance away from their nearest warehouse. It should be noted that when \(p=n\), we have the original $k$-center problem.

Formally, this problem can be stated as follows:

\textbox{Given a graph $G=(V, E)$ on a set of $n$ vertices $V = \{v_1,..., v_n\}$, a maximum number of centers $k$ and a minimum coverage constraint $p$ (where $1\leq k\leq n$ and $1\leq p\leq n$):

\vspace{0.2cm}We want to find a subset of centers $C\subseteq V$ which covers a subset of vertices $Z\subseteq V$ such that $|C|\leq k$ and $|Z|\geq p$ which minimises the objective function: $max_{v\in Z}min_{c\in C}dist(v, c)$}

\textcite{charikar_algorithms_2001} proposed a 3-approximation algorithm that has $\mathcal{O}(n^{3})$ time complexity. They note that the cost of the solution must be one of the $n^{2}$ interpoint distances in $\{d(i,j)\mid i\in V, j\in V\}$. Their method is to perform a binary search on the interpoint distances, then at each iteration evaluate a cost $r$; given a radius $r$ to check, greedily choose a center that cover the most points the most uncovered points within the distance $r$ and expand this to $3r$, marked these points as covered and repeat until $k$ centers are chosen. The search looks for the lowest radius $r$ which covers more than $p$ points. For a more detailed description of this algorithm we refer the reader the original paper by \textcite{charikar_algorithms_2001} and a later paper which discussed implementation optimisations by \textcite{schwartz_efficient_2010}.