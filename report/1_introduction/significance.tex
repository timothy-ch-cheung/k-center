$k$-center problems are relevant in solving facility location problems, as described in our parcel warehouse and hospital allocation analogies. 

In addition to these typical applications of the problem, it has been applied to in the study of computer graphics. \textcite{agarwal_structured_2003} introduced a technique to render scenes with natural lighting, applying the $k$-center problem to partition an environment map (a representation of illumination used in computer graphics) into regions which is then used to render arbitrary geometric objects. They reported better results with lower computational costs than other environment mapping and ray tracing methods at the time.

Furthermore, a variation of the $k$-center problem which adds fairness constraints has been used for fair data summarisation (\cite{kleindessner_fair_2019}). Their work focused on how standard $k$-center clustering could produce biased summaries and how fairness constraints can overcome that. They demonstrated their work by summarising a collection doctor images and specifying that the summary must have an even split between male and female doctors; their algorithm produced a fair summary while an algorithm which didn't have fairness constraints often produced summaries with an uneven split. 

The colourful $k$-center problem we study in this paper has also seen application in a novel use case. In a recent paper, \textcite{do_sphetcher_2020} showed that the colorful $k$-center problem could be applied to bioinformatics with single-cell RNA sequencing; with prior knowledge of what types of cells are in the data set, they showed it is possible to extract a small number of representative cells.