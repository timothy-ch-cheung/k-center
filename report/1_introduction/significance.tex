In addition to facility location applications of the $k$-center, it has been applied to in the study of computer graphics. \textcite{agarwal_structured_2003} introduced a technique to render scenes with natural lighting, applying the $k$-center problem to partition an environment map (a representation of illumination used in computer graphics) into regions, which are used to render arbitrary geometric objects. They reported better results with lower computational costs than other environment mapping and ray tracing methods at the time.

Furthermore, the fair $k$-center problem has been used for fair data summarisation (\cite{kleindessner_fair_2019}). They observed that $k$-center clustering could produce biased summaries and proposed fairness constraints to overcome the bias. They demonstrated their work by summarising a collection images and specifying that the summary must have an even male and female split; their algorithm produced a fair summary while an algorithm which did not have fairness constraints produced summaries with an uneven split. 

The colourful $k$-center problem has also seen application bioinformatics with single-cell RNA sequencing (\cite{do_sphetcher_2020}); with prior knowledge of what types of cells are in the data set, they showed it is possible to extract a small number of representative cells.