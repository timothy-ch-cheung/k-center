Through our statistical analysis in \cref{section:colourful_k_center_gowalla} and \cref{section:colourful_k_center_synthetic} our algorithm proved to be conclusively better than the O(1)-approximation algorithm. Across 4000 trials over 80 problem instances (Gowalla and Synthetic), Colourful PBS produces solutions with a 48.6\% lower cost than the algorithm by \citeauthor{bandyapadhyay_constant_2019}, at the expense of a 3.3-fold increase in runtime. This increase in runtime is partly due to running the O(1)-approximation algorithm as a subroutine to seed the population and time required for the local search, in particular for larger problem instances. Given more time, we would explore alternative methods to generate the population and investigate genetic operators in search for more effective ones than those designed for PBS to allow for quicker convergence.

Furthermore if we were to extend the scope of our project, we would have also liked to implement two recent algorithms for the Colourful $k$-center by \textcite{jia_fair_2020} and \textcite{anegg_technique_2020}. Gaining empirical results for these theoretical algorithms would allow us to make stronger conclusions on the performance of Colourful PBS.  