After reviewing the literature, identified four key requirements:
\begin{itemize}
    \item \textbf{Algorithm implementation:} this is the primary goal of our tool, we want to provide implementations of the algorithms described in \cref{subsection:approaches_k_center} and \cref{subsection:approaches_colourful_k_center}, to compare \gls{approx_algs} algorithms with \glspl{metaheuristic}.
    \item  \textbf{Solution Visualisation:} $k$-center literature often has few or no examples, we want our tool to show visualisations of solutions from various data sets.
    \item  \textbf{Interactive algorithm visualisation:} algorithms described in literature is often quite involved (in particular genetic algorithms), therefore we want to provide a tool to inspect an algorithm and the steps it takes to arrive at a solution.
    \item  \textbf{Accessibility:} in addition to this tool helping ourselves understand the algorithms, we also aim to aid others in understanding these algorithms therefore the tool should have minimal setup requirements.
\end{itemize}

Following the motivation for accessibility, we decided that a web application is the most suitable as it requires no installation.