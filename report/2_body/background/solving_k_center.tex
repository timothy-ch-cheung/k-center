It is known that the $k$-center is NP-hard (\cite{kariv_algorithmic_1979}), if we assume that \(P \neq NP\) then a polynomial solution does not exist. Given a problem of size \(n\), there are $\binom{n}{k}$ possible candidate solutions. To calculate the cost of a candidate would involve calculating \(n\times k\) distances. A na\"ive brute force would be to enumerate all candidate solutions, calculate the cost for each candidate and return the solution with the minimum cost. In the worst case this would take $\mathcal{O}(\binom{n}{k}\cdot n\cdot k)$, the algorithm is detailed in \cref{brute-force-k-center} for the readers' convenience. Clearly brute force methods are unfeasible for realistic use cases due to combinatorial explosion even at low values of $n$. 

Therefore the requirement of an optimal solution is relaxed, an various algorithmic paradigms are applied to create algorithms that run in polynomial time. In the following sections we will describe approximation, genetic, and randomized algorithms that solve the $k$-center problem.