From our results described in \cref{section:k_center_orlib}, we decided PBS would be a better metaheuristic than GRASP for our algorithm. We propose a series of modifications to the PBS algorithm to solve the colourful $k$-center problem. We use the same genetic operators $X_1$, $X_2$, $M_1$ and $M_2$ but we make changes to the \gls{local_search} procedure. In this section we describe our modifications and show the different genetic architectures we have considered.

It has been noted for stochastic \gls{local_search} algorithms, performing many efficient searches is better than performing a small number of complex ones (\cite{pullan_memetic_2008}). Following this intuition we designed the modified \emph{find\_pair} function to retain original time complexity of $\mathcal{O}(kn^2)$. To describe the \gls{local_search} we need to first redefine the components \emph{add\_center}, \emph{remove\_center} and \emph{find\_pair}.

As we have to account for outliers, we can no longer calculate the cost of the solution through maximising $D^1_v$ ($\forall v\in V$), therefore we amend \emph{add\_center} and \emph{remove\_center} to only update the neighbourhood structures.

%TC:ignore
\begin{minipage}{0.48\textwidth}
    %TC:ignore
\begin{algorithm}[H] 
\caption{Add Center (\cite{pullan_memetic_2008})}
\label{alg:add_center}
\begin{algorithmic}[1]
\Function{add\textunderscore center}{$f$}
    \State {cost $\gets 0$}
    \State {C $\gets C\cup \{f\}$}
    \For{$v\in V$}
        \If{$dist(f,v) < D_{v}^0$}
            \State{$D_{v}^1\gets D_{v}^0$; $F_{v}^1\gets F_{v}^0$}
            \State{$D_{v}^0\gets dist(f,v)$; $F_{v}^0\gets f$}
        \ElsIf{$dist(f,v)<D_{v}^1$}
            \State{$D_{v}^1\gets dist(f,v)$; $F_{v}^1\gets f$}
        \EndIf
        \If{$D_{v}^0>cost$}
            \State{$cost\gets D_{v}^0$}
        \EndIf
    \EndFor
\EndFunction
\State{}
\State{}
\State{}
\State{}
\State{}
\State{}
\State{}
\State{}
\State{}
\State{}
\end{algorithmic}
\end{algorithm}
%TC:endignore
\end{minipage}
\hspace{0.02\textwidth}
\begin{minipage}{0.48\textwidth}
    %TC:ignore
\begin{algorithm}[H] 
\caption{Remove Center (\cite{pullan_memetic_2008})}
\label{alg:remove_center}
\begin{algorithmic}[1]
\Function{find\textunderscore next}{$v$}
    \State{min\textunderscore cost $\gets 0$; min\textunderscore center $\gets undefined$}
    \For{$c\in C$}
        \If{$dist(v,c) <$ min\textunderscore cost}
            \State{min\textunderscore cost $\gets dist(v,c)$; min\textunderscore center $\gets c$}
        \EndIf
    \EndFor
    \State \Return {(min\textunderscore cost, min\textunderscore center)}
\EndFunction
\State{}
\Function{remove\textunderscore center}{$f$}
    \State {cost $\gets 0$}
    \State {C $\gets C\setminus \{f\}$}
    \For{$v\in V$}
        \If{$F_{v}^0=f$}
            \State{$D_{v}^0\gets D_{v}^1$; $F_{v}^0\gets F_{v}^1$}
            \State{$(D_{v}^1,F_{v}^1)\gets$ find\textunderscore next($v$)}
        \ElsIf{}
            \State{$(D_{v}^1,F_{v}^1)\gets$ find\textunderscore next($v$)}
        \EndIf
        \If{$D_{v}^0>cost$}
            \State{$cost\gets D_{v}^0$}
        \EndIf
    \EndFor
\EndFunction
\end{algorithmic}
\end{algorithm}
%TC:endignore
\end{minipage}
%TC:endignore

As we no longer get the cost of the solution as a by-product of updating the neighbourhood structures, we define a new function \emph{find\_colourful\_cost} to calculate the cost of a solution given a set of centers and constraints. The time complexity of \emph{find\_colourful\_cost} is $\mathcal{O}(n$ $log$ $n)$ due to the sorting of neighbours.

%TC:ignore
\begin{algorithm}[H] 
\caption{Find colourful cost}
\label{alg:find_cost_colourful}
\begin{algorithmic}[1]
\Function{find\textunderscore colourful\textunderscore cost}{$C$}
    \State {S $\gets\emptyset$}\Comment{The set of points which are clustered\hphantom{................................................................................}}
    \State {$t_b\gets 0$; $t_r\gets 0$}\Comment{The number blue/red points currently covered\hphantom{.................................................................}}
    \State{cost $\gets max\{dist(i,j)\mid\forall i,j\in V\}$}
    \State{$nearest$\textunderscore$vertices\gets$ sort ascending $v\in V$ by $D^1_v$}\Comment{sort the set of vertices by the distance from their nearest center}
    \For{$v\in nearest$\textunderscore$vertices$}
        \If{$t_b\geq b$ \emph{and} $t_r\geq r$}
            \State\Return{cost}
        \EndIf
        \If{$v\in B$}
            \State{$t_b\gets t_b + 1$}
        \ElsIf{$v\in R$}
            \State{$t_r\gets t_r + 1$}
        \EndIf
        \State{$cost\gets D^1_v$}
    \EndFor
    \State\Return{cost}
\EndFunction
\end{algorithmic}
\end{algorithm}
%TC:endignore
\newpage
The original \emph{find\_pair} function returned the best swap by calculating the cost of removing each center in the current solution. In PBS, the center $w$ services point $v$ with max $D^1_v$, we follow this intuition and redefine $w$ as $F^1_v$ with max $D^1_v$ such that $D^1_v\leq cost$ ($\forall v\in V$). If we were to use \emph{find\_colourful\_cost} to evaluate the best swaps in \emph{find\_pair}, the complexity of \emph{find\_pair} would be $\mathcal{O}(kn^2+n^2log$ $n)$ due recalculating the solution cost at each of the $\mathcal{O}(n)$ iterations.

Therefore instead of using the exact cost of removing a center to guide our search, we define a new heuristic based on the coverage of points rather than the cost of the solution. This change is described in \cref{alg:find_pair_colourful} lines 10-11; for each center in the solution, we calculate the number of points which will no longer be covered if that center was removed from the solution. It follows that the best swaps will involve removing the centers that provide the least coverage.

%TC:ignore
\begin{algorithm}[H] 
\caption{Find Pair (modified for colourful pbs)}
\label{alg:find_pair_colourful}
\begin{algorithmic}[1]
\Function{find\textunderscore pair'}{$w$}
    \State {coverage $\gets |V|$}\Comment{A swap causing all vertices to become uncovered is the most costly swap \hphantom{........................}}
    \State {$L\gets\emptyset$}
    \For{$i\in N_{wm}$}
        \State{add\textunderscore center'(i)}
        \State {$cost\gets$ last known cost for this individual from invoking \emph{find\_colourful\_cost}}
        \For{$c\in C$}
            \State{$M_c\gets 0$}
        \EndFor
        \For{$v\in V\setminus C$}
            \If{$min(dist(i,v),D_v^1)> cost$ \emph{and} $D^0_v\leq cost$}\Comment{Does removing $F_v^0$ cause a covered point to become uncovered?} 
                \State{$M_{F_v^0}\gets M_{F_v^0} + 1$}\Comment{If so, removing $F_v^0$ incurs a cost, add 1 to the cost of removing it}
            \EndIf
        \EndFor
        \For{$c\in C$}
            \If{$M_c=coverage$}\Comment{Add to candidate list $L$ when swaps of equal cost are found\hphantom{...........................................}}
                \State{$L\gets L\cup\{(c,i)\}$}
            \ElsIf{$M_c<C$}\Comment{Reset candidate list when a lower coverage loss is found\hphantom{.................................................}}
                \State{$L\gets \{(c,i)\}$}
                \State{$coverage\gets M_c$}
            \EndIf
        \EndFor
        \State{remove\textunderscore center'(i)}
    \EndFor
    \State\Return{select\textunderscore random($L$)}
\EndFunction
\end{algorithmic}
\end{algorithm}
%TC:endignore

In line 6 (\cref{alg:find_pair_colourful}) we define the cost as the last known cost, note we explicitly do not call \emph{find\_colourful\_cost} as that will increase the complexity above our aim of ${O}(kn^2)$. Instead, we call \emph{find\_colourful\_cost} a single time at each iteration of the \gls{local_search}.

Given these components, we are now ready to define the full \gls{local_search} in \cref{alg:colourful_pbs_local_search}. The time complexity is $\mathcal{O}(kn^3)$ which is the same complexity as the original PBS \gls{local_search}. This procedure is embedded in the original PBS framework.

%TC:ignore
\begin{algorithm}[H] 
\caption{PBS Local Search (\cite{pullan_memetic_2008})}
\label{alg:pbs_local_search}
\begin{algorithmic}[1]
\Function{local\textunderscore search}{$C,g$}
    \State{iterations$\gets 0$}
    \State{stale\textunderscore iterations $\gets 0$}
    \While{iterations $<2n$ \textbf{and} stale\textunderscore iterations $<0.1(g+1)n$}
        \State{$(cost, w)=max_{v\in V}D_v^0$ }\Comment{Find the point $w$ which is furthest away from its nearest center\hphantom{............................................}}
        \State{$(c,i)\gets$ find\textunderscore pair($w$) }
        \State{add\textunderscore center(i)}
        \State{remove\textunderscore center(c)}
        \State{iterations $\gets$ iterations $+ 1$}
        \State{(new\textunderscore cost, new\textunderscore w) $=max_{v\in V}D_v^0$ }
        \If{cost = new\textunderscore cost}
            \State{stale\textunderscore iterations $\gets$ iterations $+ 1$}
        \EndIf
    \EndWhile
\EndFunction
\end{algorithmic}
\end{algorithm}
%TC:endignore

\paragraph{Alternative selection methods}~\\
In \cref{section:pbs}, we noted PBS uses an non-standard genetic selection method. This may negatively impact the diversity of the population (since the method is somewhat similar to elitism selection), therefore we also tested roulette and tournament selection. Our experiments concluded that the original selection method was the best, our full results are reported in \cref{appendix:genetic_architectures}.

\paragraph{Seeding the initial population}~\\
In our preliminary experiments, we found instances failing to converge due to the search being stuck in local minima from poor initial starting points. The centers produced by the \emph{Ban} algorithm (\cref{section:constant_colourful_k_center}) produced centers which cover many points, therefore we use this as a subroutine to seed the first individual of the population. The result of this is better performance on our training set shown in \cref{appendix:seed_population}.

\paragraph{The full algorithm}~\\
Our full algorithm combines the PBS metaheuristic, colourful \gls{local_search} and population seeding. We refer to it as colourful PBS and it is described in \cref{alg:colourful_pbs}.

%TC:ignore
\begin{algorithm}[H] 
\caption{Colourful PBS Algorithm}
\label{alg:colourful_pbs}
\begin{algorithmic}[1]
\Procedure{PBS}{}
    \State{seed\textunderscore solution$\gets$ approx\textunderscore colourful()}\Comment{procedure from \cref{alg:approx_colourful}\hphantom{............................................................................}}
    \State {C $\gets$\{seed\textunderscore solution\}}
    \For{$i \gets 1$ to population\textunderscore size}
        \State {f $\gets$ SelectRandom(V); $C_i\gets$ LocalSearch({f})}
    \EndFor
    \Repeat
        \For{$C_i \in C$}
            \For{$C_j \in C$}
                \State{$C$ $\gets C$ $+$ LocalSearch($M_1(C_j)$)}
                \State{$C$ $\gets C$ $+$ LocalSearch($M_2(X_1 (C_i,C_j))$)}
                \State{$(S_1,S_2)\gets X_2(C_i,C_j)$}
                 \State{$C$ $\gets C$ $+$ LocalSearch($M_2(S_1)$)} $+$ LocalSearch($M_2(S_2)$)
            \EndFor    
        \EndFor
    \Until{termination condition}
\EndProcedure
\end{algorithmic}
\end{algorithm}
%TC:endignore