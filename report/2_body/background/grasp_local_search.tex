\textcite{feo_greedy_1995} introduced an algorithm framework GRASP (Greedy Randomised Adaptive Search Procedures), which has been widely cited in \gls{combinatorial_optimisation} literature. The main idea of GRASP is to perform three steps iteratively until some stopping criterion is met; first construct a randomised solution, optimise it using local search, and update the solution if it is better than the best so far.

%TC:ignore
\begin{algorithm}[H] 
\caption{GRASP (\cite{feo_greedy_1995})}
\label{alg:grasp}
\begin{algorithmic}[1]
\Require $Max$\textunderscore$Iterations$: number of iterations to perform GRASP, Seed: seed to initialise random number generator
\Ensure $Best$\textunderscore$Solution$: the solution with the best cost after $Max$\textunderscore$Iterations$ have been completed
\Procedure{GRASP}{$Max$\textunderscore$Iterations$}
    \For{$i \gets 1$ to {Max\textunderscore Iterations}}
        \State {Solution$\gets$ Greedy\textunderscore Randomised\textunderscore Construction(Seed)}
        \State {Solution$\gets$ Local\textunderscore Search(Solution)}
        \State {Update\textunderscore Solution(Solution, Best\textunderscore Solution)}
    \EndFor
    \State \Return Best\textunderscore Solution
\EndProcedure
\end{algorithmic}
\end{algorithm}
%TC:endignore

\textcite{mladenovic_solving_2003} introduced the concept of a \emph{critical vertex}; a vertex $v$ is a \emph{critical vertex} if and only if $d(v, min_{c\in C}d(v, c))=Cost(C)$, that is $v$ is a point which defines the cost of the solution $C$. \textcite{battiti_new_2017} extended this concept by counting the number of \emph{critical vertices} in a solution. They created a solution construction algorithm (Greedy Randomised Build \cref{alg:greedy_construction}) and a local search algorithm (Plateau Surfer \cref{alg:plateau_surfer_local_search}), to be embedded in the \acrshort{grasp} framework.

Greedy Randomised Build has two phases and two parameters ($\alpha$ and $\beta$). The first phase selects a $\alpha\cdot k$ initial centers (\cref{alg:greedy_construction} lines 2-7). A lower $\alpha$ causes fewer centers to be selected randomly. The second phase selects the remaining centers by iteratively generating a Restricted Candidate List (RCL) and selecting a new center from it (\cref{alg:greedy_construction} lines 8-23). The candidates for a RCL is parameterised by $\beta$, a lower $\beta$ results in a smaller and greedier RCL. The time complexity is $\mathcal{O}(nk)$, since the worst case is when we have to build $k$ RCLs. 

While \acrshort{pbs} generates initial solutions purely greedily, Greedy Randomised Build generates parameterises greediness using $\alpha$ and $\beta$. \textcite{battiti_new_2017} did not report the values of $\alpha$ and $\beta$ they used to produce their results, we conducted experiments and determined that $\alpha =0.5$ and $\beta =0.25$ were suitable (our methodology is described in \cref{appendix:grasp_param}).

%TC:ignore
\begin{algorithm}[H] 
\caption{Greedy Randomised Build (\cite{battiti_new_2017})}
\label{alg:greedy_construction}
\begin{algorithmic}[1]
\Require $\alpha$: proportion of initial centers to pick ($\alpha\in [0, 1]$), $\beta$: proportion of threshold to select RCL ($\beta\in [0, 1]$) 
\Ensure $C$: a candidate solution with $|C|=k$ centers
\Function{Greedy\textunderscore Randomised\textunderscore Construction}{$V, k, \alpha, \beta$}
    \State {$C\gets\varnothing$ }
    \State {num\textunderscore initial\textunderscore centers$\gets\alpha\cdot$ $k$}
    \For{$i \gets 1$ to num\textunderscore initial\textunderscore centers}
        \State {$f\gets$ SelectRandom($V\setminus C$)}
        \State {$C\gets C\cup\{f\}$}
    \EndFor
    \While {$|C|\leq k$}
        \State $z_{min}\gets +\infty$
        \State $z_{max}\gets -\infty$
        \For {$v_i\in V\setminus C$}
            \If {$z_{min}>Cost(C\cup \{v_i\})$}
                \State $z_{min}\gets Cost(C\cup \{v_i\})$
            \EndIf
            \If {$z_{max}>Cost(C\cup \{v_i\})$}
                \State $z_{max}\gets Cost(C\cup \{v_i\})$
            \EndIf
        \EndFor
        \State {$\mu\gets z_{min} +\beta\cdot (z_{max}-z_{min})$}\Comment{The threshold of maximum cost for candidates to be in RCL}
        \State {$RCL\gets\{v_i\in V\setminus|Cost(C\cup \{v_i\})\leq\mu\}$}
        \State {$f\gets$SelectRandom($RCL$)}
        \State {$C\gets C\cup\{f\}$}
    \EndWhile
    \State \Return $C$
\EndFunction
\end{algorithmic}
\end{algorithm}
%TC:endignore

To understand how the Plateau Surfer local search works, we explicitly define the function count\_cv (\cref{alg:count_cv}). The function count\_cv counts the number of \emph{critical vertices} of a given solution $C$ with cost $r$. A naive implemention of count\_cv would have $\mathcal{O}(kn)$ time complexity, but using the VNS data structures (detailed in \cref{section:pbs}) we can reduce the time complexity to $\mathcal{O}(n)$ . 

%TC:ignore
\begin{algorithm}[H] 
\caption{Count Critical Vertices}
\label{alg:count_cv}
\begin{algorithmic}[1]
\Require $C$: the set of centers in a solution, $r$: cost of solution, $D_i^0$: neighbourhood structure defining the distance from each point to its nearest center (described further in \cref{section:pbs})
\Ensure count: number of \emph{critical vertices}
\Function{Count\textunderscore CV}{$C,r$}
    \State {count $\gets 0$}
    \For{$v\in C$}
        \If{$D_v^0=r$}
            \State{count $\gets$ count$+1$}
        \EndIf
    \EndFor
    \State \Return count
\EndFunction
\end{algorithmic}
\end{algorithm}
%TC:endignore

%TC:ignore
\begin{algorithm}[H] 
\caption{Plateau Surfer Local Search (\cite{battiti_new_2017})}
\label{alg:plateau_surfer_local_search}
\begin{algorithmic}[1]
\Require $Max$\textunderscore$Iterations$: number of iterations to perform GRASP
\Ensure $C$: the solution locally optimised with local search
\Function{Local\textunderscore Search}{$V,C, k$}
    \Repeat
        \State {modified$\gets$false}
        \For{$v_i\in C$}
            \State {best\textunderscore flip$\gets$undefined}
            \State {best\textunderscore cv\textunderscore flip$\gets$undefined}
            \State {best\textunderscore new\textunderscore cost$\gets$ Cost($C$)}
            \State {best\textunderscore cv$\gets$ CountCV($C$, Cost($C$))}
            \For{$v_j\in V\setminus C$}
                \State {$\bar{C}\gets C\setminus \{i\} \cup \{j\}$}
                \If{Cost($\bar{C}$) < best\textunderscore new\textunderscore cost}
                    \State {best\textunderscore cv$\gets$Cost($\bar{P}$)}
                    \State {best\textunderscore flip$\gets j$}
                \ElsIf {best\textunderscore flip != undefined \textbf{and} CountCV($\bar{C}$, Cost($\bar{C}$))$<$best\textunderscore cv}
                    \State {best\textunderscore cv$\gets$ CountCV($\bar{C}$, Cost($\bar{C}$))}
                    \State {best\textunderscore flip$\gets j$}
                \EndIf
            \EndFor
            \If {best\textunderscore flip != undefined}
                \State {$C\gets C\setminus \{i\} \cup $\{best\textunderscore flip\}}
                \State {modified$\gets$true}
            \ElsIf {best\textunderscore cv\textunderscore flip != undefined}
                \State {$C\gets C\setminus \{i\} \cup $\{best\textunderscore cv\textunderscore flip\}}
                \State {modified$\gets$true}
            \EndIf
        \EndFor
    \Until{modified=false}
    \State \Return $C$
\EndFunction
\end{algorithmic}
\end{algorithm}
%TC:endignore

Plateau Surfer local search checks all swaps between centers and vertices, it will perform the swap that reduces the solution cost the most (\cref{alg:plateau_surfer_local_search} lines 11-13, 19-21). They recognise this strategy may reach local minima; in that situation, a swap which results in an equal cost but a lower number of \emph{critical vertices} is made (\cref{alg:plateau_surfer_local_search} lines 14-16, 22-24). We correct a typo from the original paper in \cref{alg:plateau_surfer_local_search} line 8, which initialised best\_cv with $\bar{C}$ instead of $C$). The time complexity is $\mathcal{O}(kn^2)$.

The total time complexity of \acrshort{grasp_ps} is $\mathcal{O}(kn^2)$, bounded by a constant number of \acrshort{grasp} iterations. The key finding that \textcite{battiti_new_2017} made was, given a solution with two equal costs, the one with fewer \emph{critical vertices} is better. In their analysis, they reported better results than VNS embedded in GRASP. However, we believe \textcite{battiti_new_2017} made an error in implementing the VNS algorithm; there are two reasons for this:
\begin{enumerate}
    \item They state "as soon as a plateau is met, Mladenovi\'{c}’s local search ends" (\cite{battiti_new_2017}) but this contradicts the original paper which states VNS will "always move to a solution of equal value" (\cite{mladenovic_solving_2003})
    \item The VNS results reported by \citeauthor{battiti_new_2017} shows poorer performance than the original VNS results
\end{enumerate}

We contacted \citeauthor{battiti_new_2017} via email to verify this, but we did not receive a reply. Nevertheless their contribution remains valuable as it defines a systematic method for searching solutions of equal cost.