As explained in \cref{section:colourful_k_center}, the Colourful $k$-center problem adds constraints based on the demographics of the vertices. Vertices are partitioned by demographic $V=\{V_1,...,V_c\}$ and have corresponding covering constraints $t_1,...,t_c$). In a similar vein to \textcite{bandyapadhyay_constant_2019}, we focus on the case where $\emph{c}=2$ to keep the analysis concise. We use a simplified notation, the class of blue vertices $B=V_1$ has the coverage constraint $b=t_1$ and the class of red vertices $R=V_2$ has the coverage constraint $r=t_2$. 

In the Colourful $k$-center we not only need to figure out which centers to choose, but also which vertices to cover. In comparison to $k$-center, which had $\mathcal{O}(\binom{n}{k}\cdot n\cdot k)$ brute force time complexity, the Colourful $k$-center has $n^2$ possible radii per candidate which makes the time complexity for the brute force algorithm $\mathcal{O}(\binom{n}{k}\cdot n^3\cdot k)$; the brute force algorithm for Colourful $k$-center is detailed in \cref{brute-force-colourful-k-center}. In this section we describe the pseudo O(1)-approximation algorithm (\cite{bandyapadhyay_constant_2019}) and our own algorithm Colourful PBS.