\textcite{bandyapadhyay_constant_2019} devised a pseudo-approximation algorithm which produces a 2-approximation using at most $k+1$ centers. They also created a second procedure which exploited some geometric properties of the colourful $k$-center problem to deal with special cases where the centers are very far apart, which combined with the pseudo-approximation to produce a $(17+\epsilon)$ true approximation. However the geometric procedure is limited to a 2D Euclidean space, due to this limitation we focus only on the pseudo-approximation algorithm.

The pseudo-approximation algorithm is comprised of three components, two linear programs (\textbf{LP1} and \textbf{LP2}) and a greedy clustering algorithm. Linear programs (\acrshort{lp}s) are formed by three components:
\begin{itemize}
    \item \textbf{Decision variables:} the variables which we want to determine the values for
    \item \textbf{Objective function:} a function which want to minimise/maximise to obtain the optimal LP solution 
    \item \textbf{Constraints:} a list of linear inequalities and equalities between the variables that a valid solution must satisfy
\end{itemize}

For a more detailed overview of linear programming we refer the reader to "Understanding and Using Linear Programming" (\cite[see][Chap.~2]{matousek_understanding_2007}).

LP1 decides whether an instance of the Colourful $k$-center problem can be solved with a cost $\rho$. To understand LP1, we need to define the concept of a ball around point $j$ of radius $\rho$; the ball $B(j,\rho)$ is the set of all points within $r$ distance of the point $j$. The decision variables for LP1 are $x$ and $z$, both of which can be fractional. Given a vertex $i\in V$, $x_i$ represents the degree that $i$ is a center and $z_i$ represents the degree that $i$ is covered by a center. For example, if $z_i=0$ then $i$ is not covered by any center and is therefore an outlier. If $x_i=1$, then vertex $i$ should be opened as a center. Constraints $(3)$ and $(4)$ of LP1 ensure that the solution covers at least $r$ and $b$ points respectively. LP1 is used to determine the optimal cost (out of $n^2$ interpoint distances), by testing a feasibility of a radius $\rho$.

\centerline{\begin{minipage}{0.8\textwidth}
    \vspace{0.5cm}
    %TC:ignore
\renewcommand{\arraystretch}{0.505}
\begin{tabularx}{\linewidth}{|lllr|}
    \hline
    & & &\tabularnewline
    \multicolumn{4}{|X|}{LP1, the feasibility LP (\cite{bandyapadhyay_constant_2019})} \tabularnewline
    \hline
    & & &\tabularnewline
    \multicolumn{4}{|X|}{Given a radius $\rho$, find a feasible solution $(x,z)$ such that:} \tabularnewline
    & & &\tabularnewline
    \hphantom{.....}& $\displaystyle \sum_{i\in B(j,\rho)} x_i\geq z_j$ & $\forall j\in V$&(1)\tabularnewline
    & & &\tabularnewline
    \hphantom{.....}& $\displaystyle \hphantom{.}\sum_{\hphantom{..}i\in V} x_i\leq k\hphantom{..}$ & &(2)\tabularnewline
    & & &\tabularnewline
    \hphantom{.....}& $\displaystyle \hphantom{.}\sum_{\hphantom{..}i\in R} z_i\geq r\hphantom{..}$ & &(3)\tabularnewline
    & & &\tabularnewline
    \hphantom{.....}& $\displaystyle \hphantom{.}\sum_{\hphantom{..}i\in B} z_i\geq b\hphantom{..}$ & &(4)\tabularnewline
    & & &\tabularnewline
    \hphantom{.....}& $z_j,x_i\in [0,1]$ & $\forall i,j\in V$ &(5)\tabularnewline
    & & &\tabularnewline
    \hline
\end{tabularx}%
\renewcommand{\arraystretch}{1}
%TC:endignore
\end{minipage}}

The greedy clustering algorithm (\cref{alg:greedy_cluster_lp}) clusters the $V$ into $|O|$ clusters with centers $C$, It creates an set of updated pairs $(\tilde{x},\tilde{z})$ from the LP1 solution $(x,z)$ which shifts the "mass" from $x_i$ to its center $x_j$ (lines 7,10-13 \cref{alg:greedy_cluster_lp}).

%TC:ignore
\begin{algorithm}[H] 
\caption{Greedy Clustering (\cite{bandyapadhyay_constant_2019})}
\label{alg:greedy_cluster_lp}
\begin{algorithmic}[1]
\Function{cluster}{$x,z,\rho$}
    \State {$C\gets\emptyset$}
    \State {$V'\gets V$}
    \While{$v'\neq\emptyset$}
        \State{$j\in V'$ be a point with maximum $z_j$}
        \State{$C\gets C\cup\{j\}$}
        \State{$\tilde{x_j}\gets min(1, \sum _{i\in B(j,\rho)}x_j)$}
        \State{$\tilde{z_j}\gets\tilde{x_j}$}
        \State{$O_j\gets B(j,2\rho)\cap V'$}
        \For{$i\neq j\in O_j$}
            \State{$\tilde{x_i}\gets 0$}
            \State{$\tilde{z_i}\gets\tilde{z_j}$}
        \EndFor
        \State{$V'\gets V'\setminus O_j$}
    \EndWhile
    \State\Return{$O,C,\tilde{x},\tilde{z}$}
\EndFunction
\end{algorithmic}
\end{algorithm}
%TC:endignore

Given clusters $O$ which contain vertices at most $2\rho$ distance away from the center, if we can pick $k$ clusters from $O$ which meet the blue/red constraints then we have a 2-approximation. \citeauthor{bandyapadhyay_constant_2019} proved while it is not always possible to select $k$ clusters which satisfy the constraints, it is always possible to select $k+1$ clusters. They formalised this procedure in LP2 which maximises the coverage of red vertices $r_i$ using centers $\tilde{x_i}$, constrained to covering at least $b$ blue vertices using $\leq k$ centers. The decision variable $\tilde{x}$ indicates which vertices to open as centers. They proved there can be at most two fractional values for $\tilde{x_i}$, which can both be rounded up for solution with at most $k+1$ centers.

\centerline{\begin{minipage}{0.8\textwidth}
    \vspace{0.5cm}
    %TC:ignore
\renewcommand{\arraystretch}{0.7}
\begin{tabularx}{\linewidth}{|lllr|}
    \hline
    & & &\tabularnewline
    \multicolumn{4}{|X|}{LP2, the simplified LP (\cite{bandyapadhyay_constant_2019})} \tabularnewline
    \hline
    & & &\tabularnewline
    \multicolumn{4}{|X|}{maximise\hphantom{.....}$\displaystyle \sum_{i\in C} r_{i}\tilde{x_i}$} \tabularnewline
    & & &\tabularnewline
    subject to & $\displaystyle \sum_{i\in C} \tilde{x_i}\geq b$ & &(1)\tabularnewline
    & & &\tabularnewline
    & $\displaystyle \sum_{i\in C} \tilde{x_i}\leq k$ & &(2)\tabularnewline
    & & &\tabularnewline
    \hphantom{.....}& $\tilde{x_i}\in [0,1]$,\hphantom{....}$\forall i\in C$ & &(3)\tabularnewline
    & & &\tabularnewline
    \hline
\end{tabularx}%
\renewcommand{\arraystretch}{1.0}
%TC:endignore
\end{minipage}}

As their research takes a purely theoretical viewpoint, they assume that $\rho$ (LP1) is 'guessed' correctly, hence $\rho=OPT$. However from a practical viewpoint, there are $n^2$ candidate costs ; a na\"{i}ve implementation would need $\mathcal{O}(n^2)$ calls to LP1. In our testing, each call to LP1 is time consuming. We show a practical implementation of the pseudo-approximation algorithm, based on a binary search, which makes $\mathcal{O}(log_2 n)$ calls to LP1.

%TC:ignore
\begin{algorithm}[H] 
\caption{Pseudo 2-Approximation Colourful $k$-center}
\label{alg:approx_colourful}
\begin{algorithmic}[1]
\Procedure{approx\textunderscore colourful}{}
    \State{weights $\gets\{dist(i,j)\mid\forall i,j\in V\}$}
    \State{cost $\gets max(weights)$}
    \State{left$\gets 0$; right$\gets |weights|$}
    \State{$(x,z)\gets$ undefined}
    \While{left $\leq$ right}
        \State{mid $\gets (left + right)/2$}
        \State{$\rho\gets$ weights[mid]}
        \State{$(x',z')\gets$ LP1($\rho$)}
        \If{$(x',z')$ is a feasible solution to LP1($\rho$)}
            \State{$(x,z)\gets(x',z')$}
            \State{right$\gets mid-1$}
        \Else
            \State{left$\gets mid+1$}
        \EndIf
    \EndWhile
    \State{$(O,C,\tilde{x},\tilde{z})\gets cluster(x,z,\rho)$}
    \State{$\tilde{x}\gets LP2$}
    \State{$C\gets \{c\in C\mid \tilde{x_c}>0\}$}
    \State\Return{$C$}
\EndProcedure
\end{algorithmic}
\end{algorithm}
%TC:endignore

Analysing the complexity is slightly more involved as we use \acrshort{lp}s. The Simplex method, developed by George Dantzig in 1947 (\cite{dantzig_origins_1990}), is popular technique to solve \acrshort{lp}s; the worse case time complexity is $\mathcal{O}(2^n)$ \parencites{klee_simplex_1972}[Chapter~5.9]{matousek_understanding_2007}. However in general this is not an issue, the simplex algorithm is shown to have a polynomial \emph{smoothed complexity} (\cite{spielman_smoothed_2001}). \emph{Smoothed complexity} combines worse case and average case complexity with slight random perturbations of the inputs which provides a more practical measure of performance. Therefore the \emph{smoothed complexity} of the pseudo-approximation algorithm is $n^{\mathcal{O}(1)}$.