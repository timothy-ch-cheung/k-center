The \emph{Sh} (\cite{hochbaum_best_1985}) and \emph{Gon} (\cite{gonzalez_clustering_1985}) algorithms give 2-approximations for the $k$-center. Both researchers independently concluded that given that $P\neq NP$, there is no hope of an approximation factor lower than two. In our paper, we focus on the \emph{Gon} algorithm.

The \emph{Gon} algorithm uses a greedy \gls{heuristic}, at each iteration the client which is furthest away from its nearest center is picked as a new center to be added to $C$ (we refer to the \gls{heuristic} as \emph{farthest first}). It arbitrarily selects a point as the first center (line 3) and repeatedly applies the \emph{farthest first} \gls{heuristic} until there are $k$ centers in the solution (line 4-8). Note the solution cost is equal to $h$ (line 5) in the final iteration.

%TC:ignore
\begin{algorithm}[H] 
\caption{\emph{Gon} algorithm (\cite{gonzalez_clustering_1985})}
\label{alg:greedy_gonzalez}
\begin{algorithmic}[1]
\Require $V=\{v_1,\ldots, v_n\}$: vertices to cluster, $k$: max number of centers
\Ensure $C$: set of cluster centers, $B$: set of clusters 
\Procedure{Approx}{$V, k$}
    \State {$B$ $\gets$ $\{V\}$}\Comment{A set of sets, $\{B_1,...,B_k\}$ representing clusters (inital value is all points in $V$)}
    \State {$C$ $\gets$ \{{$v_1$}\}}\Comment The set of centers $\{c_1,...,c_k\}$ (corresponding to the clusters they belong to)
    \For{$i \gets 1$ to $k-1$}
        \State {$h, v_x$ $\gets$ {$max\{dist(v_x, C_y)| v_x\in B_y, 1\leq y \leq i\}$}}
        \State {$B_{i+1} \gets\{v_x\}$}
        \State {$B \gets B \cup B_{i+1}                            $}
        \State {$C \gets C \cup v_x$}
        \For{$j \gets 1$ to $i$}                    
            \For{$v_t \in B_j$}
                \If{$dist(v_t, C_{i+1})\leq dist(v_t, C_j)$}
                    \State{$B_j\gets B_j\setminus{\{v_t\}}$}
                    \State{$B_{i+1}\gets B_{i+1}\cup{\{v_t\}}$}
                \EndIf
            \EndFor
        \EndFor
    \EndFor
    \State \Return {$B, C$}
\EndProcedure
\end{algorithmic}
\end{algorithm}
%TC:endignore

In the remainder of this section we describe their proof that the \emph{Gon} algorithm gives a 2-approximation. To understand the proof, we need to define the notion of a clique on a graph; a clique on a graph $G(V, E)$ is a subset $T$ of $V$ such that any two pairs of vertices in T are adjacent. In other words, $T$ forms a sub graph of $G$ which is complete. \textcite{gonzalez_clustering_1985} extends the definition of a clique to define a weighted clique of a given size. A clique T is a $(k+1)$-clique of weight $h$ if satisfies two conditions:
\begin{enumerate}
    \item The cardinality of $T$ is $k+1$ ($|T|=k+1$)
    \item Every pair of distinct vertices $(x,y)$ in $T$ are at least $h$ distance apart ($\forall _{x\in V}\forall _{y\in V\setminus \{x\}}d(x,y)\geq h$)
\end{enumerate}

\begin{lemma}\label{lemma:clique_k+1}
if there exists a clique of size $k+1$ for the set of vertices $V$ of weight $h$, then $OPT(V)\geq h$
\end{lemma} 

The proof for lemma \ref{lemma:clique_k+1} is as follows. Suppose that we have a ($k+1$)-clique $T\subseteq V$ with vertices $\{v_1, ...,v_k,v_{k+1}\}$ of weight $h$, we would like to choose $k$ centers from $V$ such that all points are covered. It is not possible to select $k$ centers such that the cost to cover all vertices in $T$ is less than $h$, since by the definition of $T$ all pairwise distinct vertices are at least $h$ distance apart. Hence $OPT(V)\geq h$.

\begin{lemma}\label{lemma:clique_non_increasing}
The sequence of distances $h$ of new centers ($v_x$) added to the partial solution $C$ is non-increasing ($h_1\geq h_2 ...\geq h_{k-1}$)
\end{lemma}

At each $i^{th}$ iteration, the new center $v_x$ added is chosen by the \emph{farthest first} \gls{heuristic} with distance $h_i$, since $v_x$ was never picked as a new center before then every other distance $h_1,...,h_{i-1}$ calculated before must be at least $h_i$.

\begin{theorem} 
the \emph{Gon} algorithm generates a solution with a cost $r$ such that $r\leq 2 * OPT(V)$
\end{theorem}

To prove this theorem we show that the \emph{Gon} algorithm produces a $(k+1)$-clique of weight $h$. At the end of iteration $k-1$, we have $k$ centers $\{C_1, ...,C_k\}$. Suppose we calculate the next potential center $v_y$ using the \emph{farthest first} heuristic, with cost $h$. Since the Graph $G$ satisfies triangular inequality, the cost of the solution is less than or equal to $2*h$. Let $T$ be $\{C_1, ...,C_k\}\cup\{v_y\}$, by lemma \ref{lemma:clique_non_increasing} it is clear that this is a $(k+1)$-clique of weight $h$. As graph $G$ satisfies triangular inequality, the solution cost $r$ satisfies $r\leq 2*h$. Therefore we have the following inequalities:
\begin{itemize}
    \item $r\leq 2*h$
    \item $2 * OPT(V)\geq 2*h$ (by scaling lemma \ref{lemma:clique_k+1} by 2) 
\end{itemize}
Combining the two inequalities we have $r\leq 2*h\leq 2 * OPT(V)$. Hence $r\leq 2 * OPT(V)$.

This algorithm iterates $k-1$ times, at each iteration it performs two operations that take $\mathcal{O}(n)$ time each - the \emph{farthest first} \gls{heuristic} (line 5) and updating the clusters (lines 9-16). Therefore the time complexity is $\mathcal{O}(nk)$.
