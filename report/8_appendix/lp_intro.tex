Linear programs (\acrshort{lp}s) are formed by three components:
\begin{itemize}
    \item \textbf{Decision variables:} the variables which we want to determine the values for
    \item \textbf{Objective function:} a function which want to minimise/maximise to obtain the optimal LP solution 
    \item \textbf{Constraints:} a list of linear inequalities and equalities between the variables that a valid solution must satisfy
\end{itemize}

To solve a linear programs, techniques such as the Simplex (\cite{dantzig_origins_1990}) or the Ellipsoid method (\cite{khachiyan_1979}) are used. We can solve \acrshort{lp}s using open source solvers such as GLPK or commercial solvers such as GUROBI or CPLEX. For a more detailed overview of linear programming we refer the reader to "Understanding and Using Linear Programming" (\cite[see][Chap.~2]{matousek_understanding_2007}).